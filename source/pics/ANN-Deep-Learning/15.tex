\documentclass{article}
\begin{document}
\centering
\begin{figure}[htbp]
    \begin{minipage}{0.20\linewidth}
        \begin{tabular}{|c|c|c|}% 通过添加 | 来表示是否需要绘制竖线
            \hline  % 在表格最上方绘制横线
            $x_1$ & $x_2$ & $x_3$ \\
            \hline  %在第一行和第二行之间绘制横线
            $x_4$ & $x_5$ & $x_6$ \\
            \hline % 在表格最下方绘制横线
            $x_7$ & $x_8$ & $x_9$ \\
            \hline
        \end{tabular}
    \end{minipage}
    *
    \begin{minipage}{0.13\linewidth}
        \begin{tabular}{|c|c|}% 通过添加 | 来表示是否需要绘制竖线
            \hline  % 在表格最上方绘制横线
            $\omega_1$ & $\omega_2$ \\
            \hline  %在第一行和第二行之间绘制横线
            $\omega_3$ & $\omega_4$ \\
            \hline
        \end{tabular}
    \end{minipage}
    +
    \begin{minipage}{0.12\linewidth}
        \begin{tabular}{|c|c|}% 通过添加 | 来表示是否需要绘制竖线
            \hline  % 在表格最上方绘制横线
            $b_1$ & $b_2$ \\
            \hline  %在第一行和第二行之间绘制横线
            $b_3$ & $b_4$ \\
            \hline
        \end{tabular}
    \end{minipage}
    =
    \begin{minipage}{0.12\linewidth}
        \begin{tabular}{|c|c|}% 通过添加 | 来表示是否需要绘制竖线
            \hline  % 在表格最上方绘制横线
            $p_1$ & $p_2$ \\
            \hline  %在第一行和第二行之间绘制横线
            $p_3$ & $p_4$ \\
            \hline
        \end{tabular}
    \end{minipage}
\end{figure}
\end{document}